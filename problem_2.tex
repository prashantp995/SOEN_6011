\documentclass[a4paper,10pt]{article}
\usepackage[utf8]{inputenc}
\usepackage{geometry}
\usepackage[T1]{fontenc}
\usepackage{url}
\def\infinity{\rotatebox{90}{8}}
 \geometry{
 a4paper,
 total={170mm,257mm},
 left=15mm,
 top=15mm,
 right=15mm,
 bottom=15mm
 }

\title{SOEN 6011 Project - Calculator-F3}
\author{Prashantkumar Patel - 40046876}

\begin{document}

\maketitle

\section {Requirements} 
\subsection \normalfont \fontsize{12}{16} {ID: REQ-1 \newline Type : Functional Requirement
\newline Priority: 1 ( 1- Highest, 5 - Lowest )
\newline Description: if the user provides input of real number x, the system should calculate and display the hyperbolic sine value for the given input.
}
\subsection \normalfont \fontsize{12}{16} {ID: REQ-2 \newline Type : Functional Requirement
\newline Priority: 1 ( 1- Highest, 5 - Lowest )
\newline Description: if the user provides input other than a number, such as special character or alphabet then function should not produce any value.
}
\section{Assumptions}
\subsection \normalfont \fontsize{12}{16} {ID: ASSUMPTION-1 \newline Description: Mathematically hyperbolic sine function has a domain of all possible real numbers. As function will be implemented on the computer using java as a programming language, there is a limitation on user input because of the datatype.
+ 1.79769313486231570E+308  is the maximum input number that the user can provide and \newline -1.79769313486231570E+308 is the minimum number that the user can provide.
}
\subsection \normalfont \fontsize{12}{16} {ID: ASSUMPTION-2 \newline Description: As the user can provide decimal numbers as input, there is a limitation on the number of decimal points considered for the calculation due to the data type used. First 10 decimal points will be considered as a significant decimal point
}
\begin{thebibliography}{9}
\bibitem{} 
\url{http://www.mathcentre.ac.uk/resources/workbooks/mathcentre/hyperbolicfunctions.pdf}
 
\bibitem{} 
\url{https://www.analyzemath.com/DomainRange/domain_range_functions.html}
 
\bibitem{} 
\url{http://functions.wolfram.com/ElementaryFunctions/Sinh/04/}
\end{thebibliography}
\end{document}
