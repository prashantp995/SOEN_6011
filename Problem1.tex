\documentclass[a4paper,10pt]{article}
\usepackage[utf8]{inputenc}
\usepackage{geometry}
\usepackage[T1]{fontenc}
\usepackage{url}
\def\infinity{\rotatebox{90}{8}}
 \geometry{
 a4paper,
 total={170mm,257mm},
 left=15mm,
 top=15mm,
 right=15mm,
 bottom=15mm
 }

\title{SOEN 6011 Project - Calculator}
\author{Prashantkumar Patel}

\begin{document}

\maketitle

{\fontsize{12}{16}\selectfont This document is shows basic understanding of hyperbolic sin function: \\} 

{\Large\textbf{1 Function:}\\}
\newline
\indent\indent
{\fontsize{12}{16}\selectfont Given functionality for calculation is  sinh(x) =   {(e\textsuperscript{x} - e\textsuperscript{-x})}{/2}\\}

{\Large\textbf{2 Definition:}\\}
\newline
\indent\indent
{\fontsize{12}{16}\selectfont The  hyperbolic  functions  are defined
 in terms of the exponential function.they  have similar
\indent\indent names to the trigonmetric functions \\}
\newline
\indent\indent
{\fontsize{12}{16}\selectfont For the given function ,e is the  base of natural log . \\}
\newline
\indent\indent
{\fontsize{12}{16}\selectfont Approximate value of the e is  2.71828\\}


{\Large\textbf{3 Domain and Co-Domain} \\}
\newline
\indent\indent
{\fontsize{12}{16}\selectfont Domain of the hyperbolic sin function is all real numbers  \\}


{\Large\textbf{4 Characteristics:}\\}
\newline
\indent\indent
{\fontsize{12}{16}\selectfont sinh(x) $\approx$ cosh (x) for large x.
sinh(x) $\approx$  -cosh (x) for large negative x \\}
\newline\indent\indent
{\fontsize{12}{16}\selectfont sinh(x) is odd function , sinh(-x) = - sinh(x) \\}
\newline\indent\indent
{\fontsize{12}{16}\selectfont The graph of sinh x
is always between the graphs e\textsuperscript{x}/2  and e\textsuperscript{-x}/2 \\}
\newline\indent\indent
{\fontsize{12}{16}\selectfont sinh(x) has period of 2$\pi$$\imath$\\}

{\Large\textbf{5 References:}\\}
\newline\indent\indent
{\fontsize{12}{12}\selectfont 1 \url{http://www.mathcentre.ac.uk/resources/workbooks/mathcentre/}\\}\indent\indent
{\fontsize{12}{12}\selectfont \url{hyperbolicfunctions.pdf}\\}
\newline\indent\indent
{\fontsize{12}{12}\selectfont 2 \url {https://www.analyzemath.com/DomainRange/domain_range_functions.html} \\}
\newline\indent\indent
{\fontsize{12}{12}\selectfont 3 \url {http://functions.wolfram.com/ElementaryFunctions/Sinh/04/} \\}
\newline\indent\indent
{\fontsize{12}{12}\selectfont 4 \url {https://reference.wolfram.com/language/ref/Sinh.html} \\}
\end{document}
